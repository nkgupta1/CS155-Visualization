\newif\ifshowsolutions
\showsolutionstrue
\documentclass{article}
\usepackage{listings}
\usepackage{amsmath}
%\usepackage{subfigure}
\usepackage{subfig}
\usepackage{amsthm}
\usepackage{amsmath}
\usepackage{amssymb}
\usepackage{graphicx}
\usepackage{mdwlist}
\usepackage[colorlinks=true]{hyperref}
\usepackage{geometry}
% \usepackage{titlesec}
\geometry{margin=1in}
\geometry{headheight=2in}
\geometry{top=2in}
\usepackage{palatino}
\usepackage{mathrsfs}
\usepackage{fancyhdr}
\usepackage{paralist}
\usepackage{todonotes}
\setlength{\marginparwidth}{2.15cm}
\usepackage{tikz}
\usetikzlibrary{positioning,shapes,backgrounds}
\usepackage{float} % Place figures where you ACTUALLY want it
\usepackage{comment} % a hack to toggle sections
\usepackage{ifthen}
\usepackage{mdframed}
\usepackage{verbatim}
\usepackage[strings]{underscore}
\usepackage{listings}
\usepackage{bbm}
\usepackage{multicol}
\setlength{\columnseprule}{0.4pt}
\rhead{}
\lhead{}

\renewcommand{\baselinestretch}{1.15}

% Shortcuts for commonly used operators
\newcommand{\E}{\mathbb{E}}
\newcommand{\Var}{\operatorname{Var}}
\newcommand{\Cov}{\operatorname{Cov}}
\newcommand{\Bias}{\operatorname{Bias}}
\DeclareMathOperator{\argmin}{arg\,min}
\DeclareMathOperator{\argmax}{arg\,max}

% do not number subsection and below
\setcounter{secnumdepth}{1}

% custom format subsection
% \titleformat*{\subsection}{\large\bfseries}

% set up the \question shortcut
\newcounter{question}[section]
\newenvironment{question}[1][]
  {\refstepcounter{question}\par\addvspace{1em}\textbf{Question~\Alph{question}\!
    \ifthenelse{\equal{#1}{}}{}{ [#1 points]}: }}
    {\par\vspace{\baselineskip}}

\newcounter{subquestion}[question]
\newenvironment{subquestion}[1][]
  {\refstepcounter{subquestion}\par\medskip\textbf{\roman{subquestion}.\!
    \ifthenelse{\equal{#1}{}}{}{ [#1 points]:}} }
  {\par\addvspace{\baselineskip}}

% \titlespacing\section{0pt}{12pt plus 2pt minus 2pt}{0pt plus 2pt minus 2pt}
% \titlespacing\subsection{0pt}{12pt plus 4pt minus 2pt}{0pt plus 2pt minus 2pt}
% \titlespacing\subsubsection{0pt}{12pt plus 4pt minus 2pt}{0pt plus 2pt minus 2pt}


\newenvironment{hint}[1][]
  {\begin{em}\textbf{Hint: }}{\end{em}}

\ifshowsolutions
  \newenvironment{solution}[1][]
    {\par\medskip \begin{mdframed}\textbf{Solution~\Alph{question}#1:} \begin{em}}
    {\end{em}\medskip\end{mdframed}\medskip}
  \newenvironment{subsolution}[1][]
    {\par\medskip \begin{mdframed}\textbf{Solution~\Alph{question}#1.\roman{subquestion}:} \begin{em}}
    {\end{em}\medskip\end{mdframed}\medskip}
\else
  \excludecomment{solution}
  \excludecomment{subsolution}
\fi

\newcommand{\boldline}[1]{\underline{\textbf{#1}}}

\chead{%
  {\vbox{%
      \vspace{2mm}
      \large
      Machine Learning \& Data Mining \hfill
      Caltech CS/CNS/EE 155 \hfill \\[1pt]
      Miniproject 3\hfill
      Released March $3^{rd}$, 2017 \\
    }
  }
}

\usepackage{amsfonts} %Mathbb Fonts
\usepackage{amsmath} %Text in Math
\usepackage{longtable} %Make multi-page tables
\usepackage{enumitem}
\usepackage{graphicx} % Import pics \includegraphics[scale=0.7]{SGD}
\graphicspath{{../graphics/}} % Change pic path
\usepackage{makecell}
\usepackage[margin=2.25cm]{caption}
\setlength{\parindent}{20pt}
\usepackage[toc,page]{appendix}
\usepackage[font=small,labelfont=bf]{caption}
\usepackage{subfig}

\begin{document}
\pagestyle{fancy}

\section{Introduction}
\medskip
\begin{itemize}

    \item \boldline{Group members} \\
    Vaibhav Anand \\
    Nikhil Gupta \\
    Michael Hashe
    
    \item \boldline{Team name} \\
    The Breakfast Club

    \item \boldline{GitHub link} \\
    \href{https://github.com/nkgupta1/CS155-Visualization}{https://github.com/nkgupta1/CS155-Visualization}
    
    \item \boldline{Division of labour} \\
    Vaibhav Anand: Matrix Factorization Visualizations, Report\\
    Nikhil Gupta: Matrix Factorization Algorithm, Report \\
    Michael Hashe: Basic Visualizations, Report

\end{itemize}

\section{Basic Visualizations}
\medskip
\subsection*{Justify your choice of visualization method}
For each of the basic visualizations, we displayed data in (normalized) histogram format. This method was suggested by the project guidelines, and is also a very clear way of comparing frequency of ratings. This method is well-suited to visualizing data involving counts (i.e., no error bars, variation, etc.), which is what we are asked to do in this section. For Part 3, we provide two graphs; the top 10 highest rated movies (all of which have only been rated 5), and the top 10 highest rated movies with at least 50 ratings (these also happen to be the top 10 highest rated movies with at least 10 ratings, so the cutoff isn't too important).
\smallskip
\subsection*{What did you observe?}
For Part 1, in which we visualized all ratings, we observed that 3's and 4's were present more frequently than would be expected in a uniform rating scheme ($\sim$ 0.2 frequency for each), 5's were present exactly as frequently as expected, and 1's and 2's were underrepresented.

For Part 2, in which we visualized the 10 most popular movies as determined from the number of rankings (Appendix \ref{appendix:popular}), we found that the most frequent ratings were 4's and 5's, both of which increased in frequency compared to the total dataset. The frequency of 1's, 2's, and 3's decreased.

For Part 3, in which we visualized the 10 highest rated movies (Appendix \ref{appendix:highest}), we found (somewhat unsurprisingly) that all ratings were 5's. When we only considered movies with sufficiently many ratings (Appendix \ref{appendix:highestpruned}), we found that the most common rating was still 5, with 4's also highly represented. Lower ratings were uncommon, with around 12$\%$ of total ratings being 3's or below.

For Part 4, in which we considered ratings for particular genres, we examined Fantasy, Sci-Fi, and War movies. Sci-Fi movies followed the same general trend as overall ratings, while both Fantasy and War movies tended to be skewed towards high ratings.
\smallskip
\subsection*{Did the results match what you would expect to see?}
For Part 1, the results matched my expectations. We would expect that better movies would be watched more, which would lead to higher ratings. 

For Part 2, the results were close to my expectations. I personally would have expected the most popular movies to have an overwhelming majority of 5's; while the most popular movies are significantly more highly rated than the general dataset, their most common rating is still a 4.

For Part 3, the results matched my expectations, both for the top 10 highest rated movies and the top 10 highest rated movies with at least 50 ratings. In the first case, I expected that there would be movies with very few ratings, and of those several would only have 1 or 2 ratings of 5 (as it turns out, there were exactly 10 of them). In the second case, it makes sense that the most highly-rated movies would have ratings near to 5 (in fact, the top ratings were closer to 4.5); as such, we would expect a sizable majority of ratings to be 5's, with very few low ratings.

For Part 4, the results matched my expectation. Given that these are 3 fairly large (and overlapping) genres, I expected that they would match both the overall ratings frequencies and the frequencies of each other. While Fantasy and War movies skewed toward higher ratings than Sci-Fi, the general patterns were very similar.
\smallskip
\subsection*{How do the ratings of the best movies compare to those of those of the most popular movies?}
As mentioned above, the ratings of the most popular movies tend to be somewhat similar to the ratings of the overall dataset, while the ratings of the highest-rated movies ($\geq$ 50 ratings) are heavily skewed towards 5's. I would have expected the ratings of these two graphs would be more similar. To explain this difference, it appears that the more popular movies are aimed towards a more general audience (i.e. Toy Story, Independence Day). In contrast, many of the most highly-rated movies are somewhat older and (in my opinion) more culturally significant (i.e. Casablanca, 12 Angry Men); if we assume that user ratings are fairly recent, then it is possible that their ratings for these movies are boosted from a targeted audience that is specifically searching for them, whereas ratings for the most popular movies are from a more representative sample of the general public.

\subsection*{How do the ratings of the three genres you chose compare to one another?}
They are very similar. This is not unexpected, as the Fantasy, Sci-Fi, and War genres overlap significantly (i.e., Star Wars). Further, all of these genres appeal to a fairly similar demographic, so we would expect the users ratings these movies to have similar rating patterns.
\pagebreak

\section{Matrix Factorization Algorithm}
\medskip
\subsection*{What parameters did you adjust and how?}
\noindent We adjusted:
\begin{itemize}
  \item the learning rate $\eta$
  \item the regularization strength $\lambda$
  \item the stopping parameter $\epsilon$
\end{itemize}
For the ways we adjusted $\eta,\lambda$, please see Appendix \ref{appendix:gridsearch}. How we adjusted $\epsilon$ is described in the next subsection*.

\subsection*{Justify your choices for the parameters and stopping criteria}
We grid searched for values of the $\eta,\lambda$ that would provide the best test error. We broke up the data into a training set including $\frac{2}{3}$ of the data and a test set containing the remaining $\frac{1}{3}$ of the data. Please see Appendix \ref{appendix:gridsearch} for results. In addition, we tried various stopping criteria and ended up settling on $\epsilon=0.003$ as the one that gave the best test error. In order to determine this, we tried different orders of magnitude (0.0001, 0.001, 0.01, 0.1) for the value and after finding the best one (0.001), increased the resolution of the search (0.001, 0.002, 0.003, 0.004, 0.005) ending up settling on a specific value (0.003). 

\subsection*{Did you make any other significant modifications or additions}
We tried various methods of training:
\begin{itemize}
  \item excluding regularization error in calculating stopping criterion
  \item including regularization error in calculating stopping criterion
  \item removing the mean from the ratings
  \item adding a bias term for individuals and movies and training on that
\end{itemize}
Each of these changes progressively made the out of sample error better.

\section{Matrix Factorization Visualization}
\medskip
\subsection*{What did you observe?}
We observed a significant correlation in the rating of a movie and one of the 2 dimensions of $\widetilde{V}$ [Figure ]. Although there was no discernible correlation between movie genre and either of the dimensions, we noticed that movies belonging from the same series (like sequels of Star Trek movies) appeared to be clustered closer to together compared to random noise [Figure ]. We also attempted to find correlations between popularity and dating of the movies and the two dimensions. The release date of a movie appeared to have none [Figure]. 

\subsection*{How do the ratings of the best movies compare to those of the most popular movies}
Popularity appeared to have a slight correlation in the same dimension and direction as rating in $\widetilde{V}$ [Figure ]. We believe this is because popularity is inherently correlated with the rating of a movie, meaning that more popular movies tend to have higher ratings, and since the correlation is stronger with $\widetilde{V}$ and rating, we believe the popularity correlation is a side effect of that.

\subsection*{How do the ratings of the three genres you chose compare to one another}
As shown in Figure [], there is no difference in the values of the two dimensions $\widetilde{V}$ and the three genres. Overall, the distribution of ratings for the three genres is similar and is explained in more depth and shown in section (2).

\subsection*{What was expected and what was surprising from the visualizations?}
We expected to see a stronger (or any) correlation between the 2 dimensions of $\widetilde{V}$ and the genres, as shown in the examples in class. Even after implementing bias and bias regularization in the matrix factorization and grid-searching to find the optimal step size and regularizations, the $E_{out}$ did not decrease below 0.446 and the visualization of $\widetilde{V}$ after performing SVD did not result in any meaningful correlations besides the average rating of movies, which is rather obvious with ML.

\subsection{Any other comparisons/observations}
While there may be correlations between features in the movies given by $\widetilde{V}$, they go beyond genres and dating, and have esoteric meaning. More detailed observations on individual genres can be seen in Figures [A-B].

Also, before we implemented bias and bias regularization in the matrix factorization, the values of $V$ for the movies were in the positive quadrant, the correlation between popularity and rating appeared to be stronger, and $E_{out}$ was higher. However, after bias, the data became more centered in both dimensions.


\section{Conclusion}
\medskip
\subsection*{Briefly summarize your main observations}
\subsection*{Did your visualizations help you to better understand the MovieLens dataset?}

\pagebreak
\begin{appendices}

\section{Basic Visualizations Data}

\subsection{Most Popular Movies}
\label{appendix:popular}
\texttt{%
  Star Wars (1977) \\
  Contact (1997) \\
  Fargo (1996) \\
  Return of the Jedi (1983) \\
  Liar Liar (1997) \\
  "English Patient, The (1996)" \\
  Scream (1996) \\
  Toy Story (1995) \\
  Air Force One (1997) \\
  Independence Day (ID4)  (1996)
}

\subsection{Highest Ratings (All Movies)}
\label{appendix:highest}
\texttt{%
  "Great Day in Harlem, A (1994)" \\
  They Made Me a Criminal (1939) \\
  Prefontaine (1997) \\
  Marlene Dietrich: Shadow and Light (1996) \\
  Star Kid (1997) \\
  "Saint of Fort Washington, The (1993)" \\
  Santa with Muscles (1996) \\
  Aiqing wansui (1994) \\
  Someone Else's America (1995) \\
  Entertaining Angels: The Dorothy Day Story (1996)
}

\subsection{Highest Ratings (\texorpdfstring{$\geq$}\ 50 Ratings)}
\label{appendix:highestpruned}
\texttt{%
  "Close Shave, A (1995)" \\
  Schindler's List (1993) \\
  "Wrong Trousers, The (1993)" \\
  Casablanca (1942) \\
  Wallace \& Gromit: The Best of Aardman Animation (1996) \\
  "Shawshank Redemption, The (1994)" \\
  Rear Window (1954) \\
  "Usual Suspects, The (1995)" \\
  Star Wars (1977) \\
  12 Angry Men (1957)
}

\pagebreak
\section{Grid Search Results}
\label{appendix:gridsearch}
\subsection{error without regularization in training (ordered by out of sample error)}

\noindent first $\frac{2}{3}$ training, last $\frac{1}{3}$ test: \\
\texttt{
  errOut = 0.643741, reg = 0.10000, eta = 0.0100, errIn = 0.267795 \\
  errOut = 0.651725, reg = 0.10000, eta = 0.0200, errIn = 0.252123 \\
  errOut = 0.661422, reg = 0.10000, eta = 0.0050, errIn = 0.286361 \\
  errOut = 0.666840, reg = 0.10000, eta = 0.0090, errIn = 0.269846 \\
  errOut = 0.670787, reg = 0.10000, eta = 0.0100, errIn = 0.250795 \\
  errOut = 0.676144, reg = 0.08000, eta = 0.0100, errIn = 0.220056 \\
  errOut = 0.676604, reg = 0.11000, eta = 0.0100, errIn = 0.283961 \\
  errOut = 0.677733, reg = 0.18000, eta = 0.0100, errIn = 0.387222 \\
  errOut = 0.678625, reg = 0.12000, eta = 0.0100, errIn = 0.301132 \\
  errOut = 0.679735, reg = 0.10000, eta = 0.0100, errIn = 0.281514 \\
  errOut = 0.682994, reg = 0.10000, eta = 0.0100, errIn = 0.262050 \\
  errOut = 0.683584, reg = 0.10000, eta = 0.0500, errIn = 0.284322 \\
  errOut = 0.683747, reg = 0.10000, eta = 0.0100, errIn = 0.259213 \\
  errOut = 0.686960, reg = 0.09000, eta = 0.0100, errIn = 0.261056 \\
  errOut = 0.692979, reg = 0.10000, eta = 0.0100, errIn = 0.258826 \\
  errOut = 0.714738, reg = 0.10000, eta = 0.0500, errIn = 0.284227 \\
  errOut = 0.731857, reg = 0.15000, eta = 0.0100, errIn = 0.362397 \\
  errOut = 0.751690, reg = 0.10000, eta = 0.1000, errIn = 0.388069 \\
  errOut = 0.752882, reg = 0.30000, eta = 0.0100, errIn = 0.468818 \\
  errOut = 0.753859, reg = 0.20000, eta = 0.0100, errIn = 0.420194 \\
  errOut = 0.761073, reg = 0.10000, eta = 0.1000, errIn = 0.394411 \\
  errOut = 0.821591, reg = 0.40000, eta = 0.0100, errIn = 0.516962 \\
  errOut = 0.852405, reg = 0.01000, eta = 0.0100, errIn = 0.152351 \\
  errOut = 0.854635, reg = 0.50000, eta = 0.0100, errIn = 0.560920 \\
  errOut = 0.918906, reg = 0.60000, eta = 0.0100, errIn = 0.620285 \\
  errOut = 0.925896, reg = 0.00010, eta = 0.0100, errIn = 0.142089 \\
  errOut = 0.931411, reg = 0.70000, eta = 0.0100, errIn = 0.681322 \\
  errOut = 0.935970, reg = 0.00100, eta = 0.0100, errIn = 0.148187 \\
  errOut = 0.943312, reg = 0.00000, eta = 0.0100, errIn = 0.142968 \\
  errOut = 1.066705, reg = 0.80000, eta = 0.0100, errIn = 0.769256 \\
  errOut = 1.250584, reg = 1.00000, eta = 0.0100, errIn = 0.956924 \\
}

\noindent first $\frac{1}{3}$ training, last $\frac{2}{3}$ test: \\
\texttt{
  errOut = 0.460330, reg = 0.10000, eta = 0.0100, errIn = 0.260879 \\
  errOut = 0.461616, reg = 0.10000, eta = 0.0050, errIn = 0.274322 \\
  errOut = 0.462888, reg = 0.10000, eta = 0.0090, errIn = 0.267259 \\
  errOut = 0.467803, reg = 0.10000, eta = 0.0200, errIn = 0.270230 \\
  errOut = 0.651215, reg = 0.01000, eta = 0.0090, errIn = 0.140637 \\
  errOut = 0.659732, reg = 0.01000, eta = 0.0100, errIn = 0.145086 \\
  errOut = 0.681749, reg = 0.01000, eta = 0.0050, errIn = 0.129947 \\
  errOut = 0.687374, reg = 0.01000, eta = 0.0200, errIn = 0.151102
}

\subsection{error with reg in training and mean removed}

\noindent first $\frac{2}{3}$ training, last $\frac{1}{3}$ test: \\
\texttt{
  errOut = 0.472064, reg = 0.10000, eta = 0.0050, errIn = 0.222142 \\
  errOut = 0.474038, reg = 0.10000, eta = 0.0100, errIn = 0.224702 \\
  errOut = 0.477569, reg = 0.10000, eta = 0.0010, errIn = 0.290618 \\
  errOut = 0.491969, reg = 0.10000, eta = 0.1000, errIn = 0.237443 \\
  errOut = 0.624262, reg = 1.00000, eta = 0.0050, errIn = 0.638081 \\
  errOut = 0.624280, reg = 1.00000, eta = 0.1000, errIn = 0.638152 \\
  errOut = 0.624281, reg = 1.00000, eta = 0.0100, errIn = 0.638098 \\
  errOut = 0.732767, reg = 0.01000, eta = 0.0050, errIn = 0.114247 \\
  errOut = 0.789952, reg = 0.01000, eta = 0.0100, errIn = 0.103601 \\
  errOut = 0.819212, reg = 0.01000, eta = 0.1000, errIn = 0.202040 \\
  errOut = 0.847534, reg = 0.00100, eta = 0.0050, errIn = 0.110945 \\
  errOut = 0.868467, reg = 0.00010, eta = 0.1000, errIn = 0.440801 \\
  errOut = 0.880642, reg = 0.00010, eta = 0.0050, errIn = 0.109889 \\
  errOut = 0.921586, reg = 0.00100, eta = 0.1000, errIn = 0.417915 \\
  errOut = 1.048777, reg = 0.00100, eta = 0.0100, errIn = 0.100087 \\
  errOut = 1.088209, reg = 0.00010, eta = 0.0100, errIn = 0.100609
}

\subsection{error with reg in training, mean removed, bias term, stopping parameter = 0.003:}

\noindent first $\frac{2}{3}$ training, last $\frac{1}{3}$ test: \\
\texttt{
  errOut = 0.446253, reg = 0.10000, eta = 0.0100, errIn = 0.242214 \\
  errOut = 0.448719, reg = 0.10000, eta = 0.0050, errIn = 0.255420 \\
  errOut = 0.478732, reg = 0.10000, eta = 0.1000, errIn = 0.230768 \\
  errOut = 0.492864, reg = 1.00000, eta = 0.0100, errIn = 0.476065 \\
  errOut = 0.494233, reg = 1.00000, eta = 0.0050, errIn = 0.475877 \\
  errOut = 0.507058, reg = 1.00000, eta = 0.1000, errIn = 0.493825 \\
  errOut = 0.594936, reg = 0.01000, eta = 0.0050, errIn = 0.150427 \\
  errOut = 0.621251, reg = 0.01000, eta = 0.0100, errIn = 0.136140 \\
  errOut = 0.798689, reg = 0.01000, eta = 0.1000, errIn = 0.207856
}


\section{Matrix Factorization Visualizations}
\label{appendix:matrixvisual}
\begin{center}
% Figure N
\begin{minipage}{0.48\linewidth}
\includegraphics[scale=0.35]{"Popularity Comparison"}
\captionof{figure}{Some here}
\end{minipage}
\hfill % spread out row
% Figure N+1
\begin{minipage}{0.48\linewidth}
\includegraphics[scale=0.35]{"Rating Comparison"}
\captionof{figure}{Some here}
\end{minipage}
% Figure N+2
\begin{minipage}{0.48\linewidth}
\includegraphics[scale=0.35]{"3-Genre Comparison"}
\captionof{figure}{Some here}
\end{minipage}
\hfill % spread out row
\begin{minipage}{0.48\linewidth}
\includegraphics[scale=0.35]{"All Genre Comparison"}
\captionof{figure}{Some here}
\end{minipage}
\begin{minipage}{0.48\linewidth}
\includegraphics[scale=0.35]{"Series: Star Trek"}
\captionof{figure}{Some here}
\end{minipage}
\hfill % spread out row
\begin{minipage}{0.48\linewidth}
\includegraphics[scale=0.35]{"Year: 1998"}
\captionof{figure}{Some here}
\end{minipage}
\end{center}

\pagebreak
\includegraphics[scale=0.26]{"genres/Genre: Action"}
\includegraphics[scale=0.26]{"genres/Genre: Adventure"}
\includegraphics[scale=0.26]{"genres/Genre: Animation"} \\ \\
\includegraphics[scale=0.26]{"genres/Genre: Childrens"}  
\includegraphics[scale=0.26]{"genres/Genre: Comedy"}
\includegraphics[scale=0.26]{"genres/Genre: Crime"} \\ \\
\includegraphics[scale=0.26]{"genres/Genre: Documentary"}
\includegraphics[scale=0.26]{"genres/Genre: Drama"}  
\includegraphics[scale=0.26]{"genres/Genre: Fantasy"} \\ \\
\includegraphics[scale=0.26]{"genres/Genre: Film-Noir"}
\includegraphics[scale=0.26]{"genres/Genre: Horror"}
\includegraphics[scale=0.26]{"genres/Genre: Musical"} \\ \\
\includegraphics[scale=0.26]{"genres/Genre: Mystery"}
\includegraphics[scale=0.26]{"genres/Genre: Romance"}
\includegraphics[scale=0.26]{"genres/Genre: Sci-Fi"} \\ \\
\includegraphics[scale=0.26]{"genres/Genre: Thriller"}  
\includegraphics[scale=0.26]{"genres/Genre: War"}
\includegraphics[scale=0.26]{"genres/Genre: Western"}
\captionof{figure}{Some here}


\end{appendices}

\end{document}